\documentclass[twocolumn,linenumbers]{aastex631}
%%\documentclass[linenumbers]{aastex631}
%%\documentclass[modern]{aastex631}
%%\documentclass[twocolumn]{aastex631}
\newcommand{\vdag}{(v)^\dagger}
\newcommand\aastex{AAS\TeX}
\newcommand\latex{La\TeX}

\usepackage{mathtools, graphicx, booktabs, apjfonts, soul}
%\addbibresource{Bibliography.bib} %For bibliography

\begin{document}

\title{Comparison of Titan's Equatorial Dunes to Lambertian Titan Surface Simulations
\footnote{Sep, 22, 2025}}

\author[0000-0002-8482-4669]{Gabriel M Steward}
\affiliation{University of Idaho, Moscow, Idaho 83844}

\author{Jason W. Barnes}
\affiliation{University of Idaho, Moscow, Idaho 83844}

\author{William Miller}
\affiliation{University of Idaho, Moscow, Idaho 83844}

\author{Shannon MacKenzie}
\affiliation{Johns Hopkins University Applied Physics Laboratory, Laurel, Maryland 20723}

\author{Others?}

\begin{abstract}

\textbf{\color{red}NOTE: Red notes are important! Do not submit the document with any of them remaining!\color{black}}

\textbf{\color{blue}NOTE: Blue notes are placeholders! Do not submit the document with any of them remaining!\color{black}}

The nature of Titan's surface is poorly understood largely due to the atmosphere's extensive interference. We simulate lambertian Titan models using the radiative transfer program SRTC++ at different surface albedos. We then compare these models to real data from Cassini VIMS (Visual and Infrared Mapping Spectrometer) of the HLS (Huygens Landing Site) and equatorial dunes. We confirm that SRTC++ produces reasonable phase functions for lambertian surfaces shrouded by Titan's atmosphere in many, but not all, situations. Furthermore, we find that Titan's dunes act as lambertian surfaces, with all identified deviations correlating to suspected SRTC++ or atmospheric model deficiencies. This is in stark contrast to the other icy moons of the Saturnian system, which exhibit notable opposition effects.

\end{abstract}

\keywords{KEYWORDS (111) --- KEYWORDS (112)}

\section{Introduction} \label{sec:intro}

Titan has one of the least understood surfaces in the entire Solar System, due largely to its thick haze-filled atmosphere that is opaque to most light. While there do exist a handful of atmospheric ``windows'' through which specific wavelengths of light can pass through relatively unimpeded \citep{Barnes2007}, only limited spectral information on the surface can be gleaned through them. Even within the windows, the thick atmosphere contaminates the relatively small amount of surface information we do receive; transmission is never perfect \citep{EsSayeh2023}. 

To combat this, we turn to radiative transfer models of Titan's atmosphere that predict the influence the atmosphere has on the received signal, allowing for true surface effects to be identified. Many such models have been created over the years, each with their own strengths and weaknesses (\cite{Griffith2012, Xu2013, Barnes2018, Corlies2021, Rannou2021}; and \cite{EsSayeh2023}, to name a few). \textbf{\color{red}Should we make sure to get ALL of these?\color{black}} These radiative transfer models depend on accurate knowledge of Titan's atmosphere, which is most well characterized at the moon's equatorial regions since that is where the Huygens lander measured the atmosphere \citep{Tomasko2008}. Many surface characterizaiton studies attempting to filter out the influence of the atmosphere have been performed in the past \citep{Buratti2006, Soderblom2009, Kazeminejad2011, Brossier2018, EsSayeh2023, Solomonidou2024}. However, the majority of them make a notable assumption: that the surface behaves as lambertian; a perfect scatterer with no directly reflected components. \cite{Buratti2006} is a notable exception. 

The lambertian assumption is considered reasonable for the equatorial regions, as the highly reflective lakes and seas of Titan are restricted to the poles \citep{Hayes2016}. However, other solid bodies in the Solar System exhibit lon-lambertian behaviors such as the prevalent opposition effect \citep{Deau2009}. All this casts doubt on the lambertian assumption: one would think Titan's surface would have non-lambertian features as well, just ones that are obfuscated by the atmosphere, particuarly since the other Saturnian moons exhibit strong opposition effects \citep{Kulyk2008}. 

In this paper, we seek to demonstrate the degree to which Titan's equatorial dunes exhibit non-lambertian behavior. We compare lambertian simulations of Titan with real observations of the dunes, identifying notable differences.

The vast majority of observations of Titan's surface have been done by spacecraft visiting Saturn, with the most high-quality data coming from the Cassini mission. As such, many images of Titan's surface are taken at unusual viewing geometries. This is quite useful, as it allows characterization of Titan's surface from a wide variety of orientations, making it far easier to determine exactly how non-lambertian a terrain is. Unfortunately, most current radiative transfer models applicable to Titan either assume a plane parallel atmosphere in their calculations \citep{Griffith2012, EsSayeh2023}, don't consider angle at all \citep{Rannou2021}, or use a spherical approximation \citep{Corlies2021}. Thus, all these lose accuracy the further the viewing geometry is from direct illumination, and would miss potential non-lambertian effects. To gain the useful information contained within observations at non-ideal viewing gometries, the spherical nature of Titan's atmosphere must be considered. Thus, to create our lambertian models, we use SRTC++ (Spherical Radiative Transfer in C++), a radiative transfer code tailored to model Titan in full spherical geometry at the infrared wavelengths available to Cassini's VIMS (Visual and Infrared Mapping Spectrometer) instrument \citep{Barnes2018}. Other spherical models do exist \citep{Xu2013}, but SRTC++ was chosen due to familiarity with the code.

Equipped with a spherical radiative transfer model and atmospheric characterization from Huygens, it is now possible to compare models with reality on a scale covering the entire Cassini mission. As the equatorial regions are the best characterized atmospherically, we chose to examine the terrain there. While we examined multiple terrain types, it was determined that only the equatorial dunes had a sufficient number of reliable observations to draw meaningful conclusions from. We also hand-picked observations from the Huygens Landing Site (HLS) for the sake of comparison, even though that location was only viewed at a limited number of viewing angles. For both the dunes and HLS we compiled VIMS observations from across the entire Cassini mission over all available viewing gometries, limiting observations to only those judged to be of sufficient quality.

This analysis serves dual purposes--to qualitatively validate the SRTC++ simulation against real data, and to identify deviations from lambertian behavior on Titan's surface. To accomplish this, first we outline improvments made to SRTC++ in \textbf{Model Methods} and report on those changes in \textbf{Model Results}. We describe the procedure by which we gathered our Titan data in \textbf{Observations and Data}, compare reality to simulation in \textbf{Model vs Data Comparison}, and end with the \textbf{Summary and Conclusion}.

\textbf{\color{red}NOTE: Be sure to redo this last introduction paragraph when the paper is done to match the final outline.\color{black}}

\section{Model Methods} \label{sec:model}

\textbf{\color{blue}METHODS: Jason's Section. Brief summary of SRTC++, citing the previous paper for more details. Describe new SRTC++ modules used, notably Abosrption and the switch to Doose atmosphere. There should be a comparison figure to note the differences between the two. Mention the new integration method.\color{black}}

\textbf{\color{blue}Figs: comparison between SRTC++ versions.\color{black}}

\textbf{\color{blue}Will be done by Jason \color{black}}

\section{Model Results} \label{sec:mresults}

Four our work, we used SRTC++ to create three different models of a uniformly lambertian Titan, each run with a different surface albedo: 0.0, 0.1, and 0.2. Besides the albedo, the simulations were set up with identical parameters, as shown in \textbf{\color{blue}Figure 1\color{black}}. The virtual detectors were set 10,000 km away from the lambertian Titan, separated by 5 degrees and completely encircling the moon. Output was generated at all eight VIMS IR wavelength windows for Titan's atmosphere: 0.93, 1.08, 1.27, 1.59, 2.01, 2.69, 2.79, and 5.00 $\mu$m.

\begin{figure}[htbp]
\includegraphics[scale = 0.5]{SRTCLayout.pdf}
\centering
\caption{Layout of our SRTC++ simulations. Distances not to scale. Detectors all equidistant from Titan and angular separation is the same for each one. The yellow arrow represents "photon packets" being shot at Titan. Note that it does not interact with the detector it passes through.}
\label{fig:1}
\end{figure}

\textbf{\color{blue}Figures 2-4\color{black}} show the results of the simulations at 0.0, 0.1, and 0.2 albedo respectively, colored with 5.00 $\mu$m as red, 2.01 $\mu$m as green, and 1.27 $\mu$m as blue, as done in \cite{Barnes2005}. The results of 0.1 are closest to the ``greenish'' pictures of Titan in \cite{Barnes2005}, though naturally without any terrain variation. 

\begin{figure}[htbp]
\includegraphics[scale = 0.4]{LambertianSimA0.pdf}
\centering
\caption{\textbf{\color{red} NOT FINAL \color{black}}Simulation results for a lambertian Titan with 0.0 albedo, colored with 5, 2 and 1.3 $\mu$m mapped to red, green, and blue respectively. Left image is viewed at 0$^{\circ}$  from the incidence angle. Right four images are at  35$^{\circ}$, 90$^{\circ}$, 120$^{\circ}$, and 180$^{\circ}$ in left to right then top to bottom order. \textbf{\color{red}[An animating version of the figure will exist in places that support it. The large left panel will hold the animating image, the right panels will remain static for comparisons] \color{black}}}
\label{fig:2}
\end{figure}

\begin{figure}[htbp]
\includegraphics[scale = 0.4]{LambertianSim.pdf}
\centering
\caption{\textbf{\color{red} NOT FINAL \color{black}}Same as Figure 2 but for 0.1 Albedo.}
\label{fig:3}
\end{figure}

\begin{figure}[htbp]
\includegraphics[scale = 0.4]{LambertianSimA2.pdf}
\centering
\caption{\textbf{\color{red} NOT FINAL \color{black}}Same as Figure 2 but for 0.2 Albedo.}
\label{fig:4}
\end{figure}

The simulations are largely as expected for a lambertian sphere obscured by a thick atmosphere. We can see what a lambertian sphere should be without an atmosphere from Fig 9 in \cite{Pont2007}, and the 0.0 albedo simulation shows us the effect of the atmosphere without any signal coming from the surface. Both effects should be active in the 0.1 and 0.2 albedo simulations, and that is in fact what we see in the directly illuminated view: a greenish center where there's minimal atmospheric influence that slowly fades off as the observation angle becomes steeper, at which point the atmosphere takes over with its bluish hue. Note that the coloration is entirely due to atmospheric effects; bluer light is scattered in the atmosphere more readily than redder light. In the raw data the blue channel is the largest even in the center at 0.1 albedo, it is merely the choice of color balancing that gives Titan its greenish hue; a sensible choice as it allows us readily identify places with more surface signal than atmospheric. The center of the disc, when viewed from head on, is rather uniform with little variation, as it should be. 

Views other than direct illumination also behave as expected; the edges are always atmospherically dominated, becoming bluer. Light is more likely to be forward scattered than backscattered in Titan's atmosphere \citep{GarcaMuoz2017, Cooper2025}, so the closer the Sun gets to behind Titan, the brighter the edges become. A lambertian surface without an atmosphere would have no light coming from anywhere that was not directly illuminated, but we see a small amount of light released from beyond the terminator due to atmospheric scattering. We can also see the shadow Titan casts on its own atmosphere in the profile views as the signal gets abrubtly diminished near the top and bottom of the moon, but the atmosphere above this shadow remains illuminated. This effect is known to happen on Earth as well and is what results in the colors of twilight \citep{Lynch2004}. 

Notably the ``eclipse'' view at 180 degrees is functionally identical in all simulations, which it should be as the surface has little to no influence on the signal at this viewing gometry. There is most certainly interesting science to be done with the simulation at this angle to probe the atmosphere, but that is beyond the scope of this paper.

The simulation results notably appear slightly grainy. This is due to the Monte-Carlo nature of SRTC++. If we were to run the simulation for longer, the S/N (signal to noise) would increase and the overall image result would become smoother. However, given as we end up averaging numerous pixels together for this work, such effort would be wasted.
 
We ultimately seek to compare these simluations with real data. And while we can qualitatively see that real Titan images do have similar coloration and limb effects \citep{Barnes2005}, the fact that real Titan has numerous different terrain types interferes with any more robust conclusions. As such, we turn these visual models into phase funciton models. These phase function models can, given a specific viewing goemetry, predict how a lambertian Titan would appear at that geometry, which can then be compared to real data. 

To construct the phase function models, every model pixel is sorted into bins based on viewing geometry and then their values are averaged. The bins are separated by 5 degree increments in incidence, emission, and azimuth angle, defined as shown in \textbf{\color{blue}Figure 5\color{black}}, with 0$^{\circ}$  azimuth being the forward scattering direction, and 180$^{\circ}$  being the backscattering. We use these angles instead of the more standard bidirectional reflectance distribution function (BRDF) as that uses four variables to describe the situation, while our models only need three. With 3 different albedos and 8 wavelengths, we produce a total of 24 phase function models ready to be compared with similar phase functions compiled from real VIMS data.

\begin{figure}[htbp]
\includegraphics[scale = 0.55]{VGDisplayLabel.pdf}
\centering
\caption{Viewing Geometry Angles used in this paper. Incidence angle is represented by ``i'', emission angle by ``e'', and azimuth angle by ``$\phi$.'' The arrows trace out a path from the sun (yellow circle), to some arbitrary spot on Titan's surface, to a detector (purple diamond).}
\label{fig:5}
\end{figure}

\section{Observations and Data} \label{sec:observe}

Cassini performed over a hundred separate flybys of Titan during its mission \citep{Seal2017}, and most of those flybys have observaitons from Cassini VIMS. Viewing geometries on any single flyby are genreally limited in scope, as the spacecraft itself could only examine geometries it personally encountered. Thus, in order to gain a proper understanding of the surface of Titan at a wide range of viewing geometries, observations from as many flybys as possible should be used. 

The primary obstacle in properly using all the data is the sheer amount in play; over a hunded flybys, tens of thousands of individual observations, and in each of those hundreds of spectels each with hundreds more individual values associated with them. If we wished to make a single global model, this would not be an issue, as an algorithm could easily ingest everything. However, simple inspection of VIMS images reveals different terrains have different albedos and our simulations above clearly show how much the appearance of the surface changes with albedo. Averaging all terrain types together could smooth out non-lambertian behavior, or allow a single non-lambertian terrain type to contaminate all the others. 

As such, we need to limit our data to single terrain types. We also chose to focus on the equatorial regions between 30$^{\circ}$  and -30$^{\circ}$ latitude since that's where Huygens sampled the atmosphere \citep{Tomasko2008}, making this region the best characterized. We also want terrain types that are viewed from a wide variety of viewing gometries with a large number of reliable observations. We judged that only Titan's dunes met all these criteria. (The equatorial bright terrain was considered as well, but the results we obtained from it were inconsistent and unreliable. See the \textbf{\color{blue}appendix\color{black}} for more information.) In pursuit of this goal, we created a raster mask of Titan's equatorial dunes \textbf{\color{blue}(Figure 6)\color{black}}.

\begin{figure}[htbp]
\includegraphics[scale = 0.25]{TitanSurfaceMaskDunes.pdf}
\centering
\caption{Titan equatorial surface terrain mask for dunes, as informed by \cite{Lopes2020} mixed with VIMS observations. Black areas are ``Null'' pixels as outlined in the text, while the red are dunes. Almost all of the dunes on Titan are included in this mask, though there are a few regions outside the 30$^{\circ}$  and -30$^{\circ}$ latitude boundaries.}
\label{fig:6}
\end{figure}

The creation of the mask began with the Titan terrain map created by \cite{Lopes2020} using radar data. VIMS observations, which are taken in infrared, often don't match the radar observations \citep{Soderblom2007}, but tend to agree in the bulk of major features; most notably (and importantly for this paper) the dunes have good agreement on a global scale.

The resolution in question for the mask is one pixel per degree on Titan's surface; 181 in latitude and 360 in longitude. The radar map was scaled down to reduce it to this resolution. Any pixel that was not clearly or nearly a solid color was replaced with a ``Null'' pixel; one where we would not harvest data from when using the mask to identify terrain. We erred on the side of caution, more likely to assign ``Null'' to a pixel than not. Any pixels of different terrains that were touching were makred ``Null'' as well to avoid contamination. After this we manually removed some areas that notably did not match VIMS data, were too small to be of use, or were known to have different spectral characteristics than other terrains given the same classification. Hotei Regio, Tui Regio, and Southern Xanadu were notable manual exclusions. The original mask had many different terrain types, but we reduced all non-dune terrains to ``Null'' for this paper.

In addition to the terrain classification in \textbf{\color{blue}Figure 6\color{black}} the mask also has a version with a hidden data point: each pixel records its distance to the nearest ``Null'' pixel in km along Titan's surface. This allows the mask to be refined: pixels that are close to ``Null'' pixels can be excluded as likely to have contamination from pointing errors in the VIMS data, which are known to occur \citep{Barnes2008}. 

VIMS observations come in the form of ``cub'' files. Our investigation procedure for these files begins with a basic database search; in our specific case, looking for any cubs that have spectels in the equatorial regions between 30 and -30 latitude, and also have spectels of 25 km ground resolution or lower. The database we are using has already filtered out clearly erroneous files, as well as so called ``noodle'' images which are only a handful of spectels in diameter. 

Once the cubs are identified, they are ingested and each individual spectel examined to see if it is satisfactory according to provided restrictions; namely, the latitude and ground resolution limitaitons. If a spectel passes the test, it is added to a list. This list can be made with or without refering to the mask. When the mask is used, only spectels marked as dunes are cataloged. The spectels are then judged based on their proximity to a ``Null'' pixel on the mask. Two options exist for this: setting a minimum distance from a ``Null'' pixel that will be accepted, or setting an allowed maximum ratio of ground resolution to ``Null'' pixel distance. For this work, we had a minimum ``Null'' distance of 50 km and a maximum ratio of ground resolution to ``Null'' pixel distance of 1/4.

When the mask is not used, ``Null'' values can be accepted. This is helpful when wanting to look at areas in or near ``Null'' pixels, such as the Huygens Landing Site. 

We then bin the list of spectels and create a phase function out of them in a process identical to that described in \textbf{\color{blue}Model Results\color{black}}. Unlike other investigations, we do not subtract off atmospheric contributions as our simulations are complete with atmospheric contributions. This saves us considerable processing difficulties.

There are a few limitaitons to the created phase functions. The primary limitation is that certain viewing angles, usually at the extreme ends of allowed values, do not exist since Cassini was never in those positions. Particuarly high resolution cubes can reveal small details not visible in most views and thus are not reflected in the mask, such as a handful of observations that can see interdune areas \citep{Barnes2008}. These small details need not match the behavior of the terrain they are surrounded by, and could conceivably offset the final model. For larger data sets, such as the dunes, situations like this are likely to be overcome by the sheer number of data points available at lower resolution. There is also no check for interfering clouds at this time. 

In addition to the dunes, we also made a phase function for the Huygens Landing Site, as our atmosphere models ultimately draw from observations made by the Huygens lander \citep{Tomasko2008}. We performed a database search similarly to how we made the other models, but we also went in manually, cleaning up any situations where Cassini reported the wrong latitude and longitude for the spectels, ensuring that the data was devoid of any contamination; something we only had to do since the Huygens Landing Site was such a small area on the boundary of multiple terrain types. Unfortunately, there were relatively few viewing angles of the HLS despite it being among the most observed locaitons on Titan, which is why examining the dunes as a whole was necessary. In order to counteract this lack of data for both the dunes and the HLS, we performed tetrahedral interpolation to fill in as many observation angles as possible, using the PyVista package \citep{Sullivan2019}.

\section{Model vs Data Comparison} \label{sec:compare}

\subsection{Huygens Landing Site} \label{sec:huygensC}

We choose to examine the HLS first as its data are simpler, though it covers significantly fewer viewing gometries than the dunes. As three-dimensional data are hard to visualize, we opt to plot individual ``skewers'' of data at a time, fixing two of the three viewing geometry angles while allowing the third to vary. We do this for all phase functions, be they models or real data. We plot select skewers of the three models and HLS data in \textbf{\color{blue}Figures 7-9\color{black}}. 

\begin{figure*}[tbp]
\includegraphics[scale = 0.4]{HLSInciSkewer.pdf}
\centering
\caption{Incidence angle skewer: incidence angle versus average I/F comparing phase function models with HLS data with fixed emission and azimuth angles. Showcases all eight wavelengths arranged from shortest to longest, with the central area occupied by a geometry diagram to illustrate the exact situation plotted. Model lines are monochromatic, HLS data are purple. Places with direct HLS observations have dots plotted overtop of the lines, larger dots meaning more observations. Places on the HLS lines without dots are interpolated values. Note that the vertical axis scales with the data, not all wavelengths produce the same average I/F scale.}
\label{fig:7}
\end{figure*}

\begin{figure*}[htbp]
\includegraphics[scale = 0.4]{HLSEmisSkewer.pdf}
\centering
\caption{Same as \textbf{\color{blue}Figure 7\color{black}} but is instead a skewer through the emission angle, with incidence and azimuth held fixed. }
\label{fig:8}
\end{figure*}

\begin{figure*}[htbp]
\includegraphics[scale = 0.4]{HLSAzimuthSkewer.pdf}
\centering
\caption{Same as \textbf{\color{blue}Figure 7\color{black}} but is instead a skewer through the azimuth angle, with incidence and emission held fixed.}
\label{fig:9}
\end{figure*}

Ignoring the HLS for a moment, the phase function models showcase a distinct shifting of behavior across the various wavelengths. The shorter wavelengths have rougher lines, while the longer ones are smoother. This is due to the general trend of the atamosphere having higher optical depth at shorter wavelengths \citep{EsSayeh2023}, allowing for more scattering events in more directions, which in turn mean more simulation time would be required to smooth out all those directions. 

We see characteristic behavioral differences in phase function behavior at the different wavelengths in \textbf{\color{blue}Figure 7\color{black}} and \textbf{\color{blue}Figure 8\color{black}}, with the general shape of emission skewers in particular changing drastically between 0.93 and 5.00 $\mu$m. When skewering across incidence or emission, this is common; though when skewering across azimuth the general shape is usually similar in all wavelengths, including the skewer shown in \textbf{\color{blue}Figure 9\color{black}}. Keep in mind that across these three figures only a small amount of the total models are shown, behavior can change significantly when skewering at different locations, though the transitions are always relatively smooth. These particular skewers were chosen because they are the places with the best HLS data visualizations.

The HLS data are very well behaved for the most part, with data that interpolates smoothly and forms lines that correspond rather well to the models. In 1.27 and 2.01 $\mu$m the data lines up almost perfectly with the 0.1 albedo prediction, while the others sit somewhere between 0.0 and 0.1. 1.08 $\mu$m hugs the 0.0 albedo line, and of all the wavelengths it shows the largest variation, most notable in \textbf{\color{blue}Figure 7\color{black}} here, but the variation is also visible in other views not shown. We are unsure why 1.08 $\mu$m in particular does this; we expect 0.93 and 1.08 data to be variable due to the large atmospheric contributions, but not for them to be much different from each other. Perhaps this is a feature of the HLS itself?

Even considering 1.08 $\mu$m's quirks, the HLS appears consistent with a lambertian surface in all the viewing geometries we have access to, which admittedly is not very many. We chose the best skewers we could for \textbf{\color{blue}Figures 7-9\color{black}}, and even the azimuth skewer only covers about a third of the available values. Nothing probes the higher incidence and emission angles, and the extremes are aren't even approached. This tells us nothing about SRTC++'s reliability beyond angles that are already probed well by other radiative transfer models. Thus, we move to a much larger data set: the dunes.

\subsection{Equatorial Dunes} \label{sec:dunesC}

Similarly to the HLS data, we plot a variety of select skewers for the dunes in \textbf{\color{blue}Figures 10-12\color{black}}. We plot more than three as the dunes data cover a far wider range of viewing gometries. 

\begin{figure*}[htbp]
\includegraphics[scale = 0.4]{DunesISkewer1.pdf}
\centering
\caption{Incidence angle skewer: incidence angle versus average I/F comparing phase function models with dunes data with fixed emission and azimuth angles. Showcases all eight wavelengths arranged from shortest to longest, with the central area occupied by a geometry diagram to illustrate the exact situation plotted. Model lines are monochromatic, dunes data are brown. Places with direct dunes observations have dots plotted overtop of the lines, larger dots meaning more observations. Places on the dunes lines without dots are interpolated values. Note that the vertical axis scales with the data, not all wavelengths produce the same average I/F scale.}
\label{fig:10}
\end{figure*}

\begin{figure*}[htbp]
\includegraphics[scale = 0.4]{DunesISkewer2.pdf}
\centering
\caption{Same as \textbf{\color{blue}Figure 10\color{black}} but at different fixed emission and azimuth angles.}
\label{fig:11}
\end{figure*}

\begin{figure*}[htbp]
\includegraphics[scale = 0.4]{DunesISkewer3.pdf}
\centering
\caption{Same as \textbf{\color{blue}Figure 10\color{black}} but at different fixed emission and azimuth angles.}
\label{fig:12}
\end{figure*}

Beginning with incidence angle skewers, we find that most skewers share a similar shape, visible in both \textbf{\color{blue}Figure 10 and Figure 11\color{black}}. Higher incidence angles have lower I/F across the board, the exception being skewers with high emission and low azimuth, but the dunes have no data at those viewing geometries so we ignore them. The dune data match the shape of the models very well in these views, though incidences higher than 80$^{\circ}$  should be taken with a grain of salt as there is very little data at those extreme angles, though the points that do exist behave as expected in these views. 

Curiously, the 2 $\mu$m has a very noticable shift in retrieved albedo for the dunes between \textbf{\color{blue}Figure 10 and Figure 11\color{black}}, going from hugging the 0.1 albedo model to hanging somewhere around 0.05 albedo. 1.27 $\mu$m may do this as well, though it is harder to tell as the gap between 0.1 and 0.0 albedo models is smaller in that window and the data are less consistent. We will return to this effect when examining the emission skewers, as the effect is far more noticable there.

Most incidence skewers not shown look like \textbf{\color{blue}Figure 10 and Figure 11\color{black}} in the phase function models, excepting those where there is no data. The dunes data usually match the models in shape rather well, indicating lambertian behavior when incidence is considered, with minor and momentary deviations for the most part. However, there are exceptions, which happen at high emission and high azimuth, exemplified by \textbf{\color{blue}Figure 12\color{black}}. Here, the 2.01, 2.69, and 2.79 $\mu$m windows do not have slopes that match the models. Unfortunately, there isn't a very large spread of points at this view, so this anomaly is hard to draw things from. We will return to this, however, when examining the emission skewers. 

\textbf{\color{blue}Figure 12\color{black}} also makes clear a physical impossibility that sometimes crops up in incidence slices: while the 0.93 and 1.08 $\mu$m windows have dune phase function shapes that match the models', the retreived albedo is below 0.0, an impossibility. This may be because SRTC++ does not account for Rayleigh scattering, which is most relevant at short wavelengths \citep{EsSayeh2023}. Intuitively, one would expect adding Rayleigh scattering to make the model brighter, not dimmer, but subtle effects could be at play--to fully answer this question we would have to implement Rayleigh scattering. 

\begin{figure*}[htbp]
\includegraphics[scale = 0.4]{DunesESkewer1.pdf}
\centering
\caption{Same as \textbf{\color{blue}Figure 11\color{black}} but is instead a skewer through the emission angle, with incidence and azimuth held fixed.}
\label{fig:13}
\end{figure*}

\begin{figure*}[htbp]
\includegraphics[scale = 0.4]{DunesESkewer2.pdf}
\centering
\caption{Same as \textbf{\color{blue}Figure 13\color{black}} but at different fixed incidence and azimuth angles.}
\label{fig:14}
\end{figure*}

\begin{figure*}[htbp]
\includegraphics[scale = 0.4]{DunesESkewer3.pdf}
\centering
\caption{Same as \textbf{\color{blue}Figure 13\color{black}} but at different fixed incidence and azimuth angles.}
\label{fig:15}
\end{figure*}

The emission skewers for the phase function models generally show gradual increases in brightness with larger emission, though 5 $\mu$m is a notable exception, along with low incidence and high azimuth views. Selected views can be seen in \textbf{\color{blue}Figures 13-15\color{black}}. \textbf{\color{blue}Figure 13 and Figure 14\color{black}} clearly show that for low emission angles, up to around 40$^{\circ}$, the data match the models pretty well. However, at higher emission angles, this fails: \textbf{\color{blue}Figure 13\color{black}} shows the 0.93 - 1.59 $\mu$m windows dipping below the 0.0 albedo model at higher emission, while 2.01 - 2.79 $\mu$m do the opposite and sharply tick above the models. The dramatic brightening of 2.01 - 2.79 $\mu$m is just as strong if not stronger in \textbf{\color{blue}Figure 14\color{black}} showing azimuthal independence of this effect. However, the 0.93 - 1.59 $\mu$m dimming below the 0.0 albedo model is not as strong, and doesn't clearly exist at the 1.27 $\mu$m and 1.59 $\mu$m windows. The dimming effect vanishes entirely in \textbf{\color{blue}Figure 15\color{black}}, while the brightening effect remains. (Notably, \textbf{\color{blue}Figure 15\color{black}} has few data points at high emission, but the fact that it follows the pattern laid out in other skewers gives us confidence that its behavior is not just an interpolation artifact). These brightening and darkening effects at higher emission completely explain the inconsistent retrieved albedo and values below 0.0 albedo noted in the incidence skewers; these deviations are emission dependent. Furthermore, the fact that brightening only occurs in 2.01 - 2.79 $\mu$m windows and these are the only windows seen not matching the models in \textbf{\color{blue}Figure 12\color{black}} lends further credence that \textbf{\color{blue}Figure 12's\color{black}} deviations are not just a bad series of observations.  

So, how can we explain these discrepancies? There are two primary explanations. First, the model is not accurately accounting for some atmospheric effect at high emission. \textbf{\color{red}RESEARCH: what parts of the atmosphere could be influencing this? What papers would be relevant? This will take significant time to sort out.\color{black}}

The second option is that we're seeing a true nonlambertian effect from the surface. However, this seems unlikely, as the brightening effect appears azimuthally agnostic: no matter what azimuth we point at, the effect exists where we have enough data to see it, and a forward-scattering or backscattering effect would be particuarly focused at 0$^{\circ}$  or 180$^{\circ}$, not present everywhere. Topogrpahy oriented explanations also seem unlikely, since the effect begins around 40$^{\circ}$ and only extreme emission angles should be greatly influenced by topography. Not to mention the fact that Titan is rather smooth, topographically speaking, with only a couple km elevation variation on the surface \citep{Corlies2017}.

\begin{figure*}[htbp]
\includegraphics[scale = 0.4]{DunesASkewer1.pdf}
\centering
\caption{Same as \textbf{\color{blue}Figure 11\color{black}} but is instead a skewer through the azimuth angle, with incidence and emission held fixed.}
\label{fig:16}
\end{figure*}

\begin{figure*}[htbp]
\includegraphics[scale = 0.4]{DunesASkewer2.pdf}
\centering
\caption{Same as \textbf{\color{blue}Figure 16\color{black}} but at different fixed incidence and emission angles.}
\label{fig:17}
\end{figure*}

\begin{figure*}[htbp]
\includegraphics[scale = 0.4]{DunesASkewer3.pdf}
\centering
\caption{Same as \textbf{\color{blue}Figure 16\color{black}} but at different fixed incidence and emission angles.}
\label{fig:18}
\end{figure*}
 
Despite this clear deviation at high emission, the model and data still match remarkably well. Nowhere is this easier shown than the azimuth skewers, which unlike the incidence and emission skewers, regularly have data in large numbers and data points covering the entire range. And in most situtaitons, the result is flat, such as in \textbf{\color{blue}Figure 16 and Figure 17\color{black}}.

\textbf{\color{blue}Figure 16\color{black}} importantly demonstrates in its short wavelength windows that we can still have points below 0.0 albedo even at low emission angles, not just high ones; indicating that there are probably at least two effects influencing this result. However, dark as the data are, they are all flat and uniform, as the models inidcate it should be. Most azimuth views are completely flat in both model and real data. 

The exception to the flatness is when both incidence and emission are high at the same time, at which point the models predict low azimuth should be brighter than the other azimuths, as can be seen in \textbf{\color{blue}Figure 18\color{black}}. Unfortunately, we have very few observations in this region of the phase function, and the interpolation is rather suspect as there aren't other regions with similar behavior as was the case with the emission skewers. \textbf{\color{blue}Figure 18\color{black}}'s interpolation still shows an uptick at low azimuth despite this, which at least suggests the behavior is plausibly accurate. One saving grace is that at such angles, the effects of the atmosphere take over and almost drown out the surface effects, so the dunes themselves are unlikely to have much effect on real observations in the first place. Note that in \textbf{\color{blue}Figure 18\color{black}} the different albedo values are all very closely clustered together and nearly identical in shape, corroborating this thought. Even though the 5 $\mu$m window seems to be spread out over albedo, this is merely due to the fact that all the models report very dim I/F values, differing only by 0.05 overall. 

In the end, where does this leave us? With the shapes of the dunes data matching the phase function models so closely in incidence and azimuth, it seems reasonable to conclude that Titan acts as a lambertian surface. The primary evidence for nonlambertian behavior comes from the emission skewers, which does not appear to have an azimuthal dependence and is likely an atmospheric effect not accounted for in the model, rather than a true non-lambertian phase function. The other minor deviation in \textbf{\color{blue}Figure 12\color{black}} correlates directly with the windows brightening in emisison skewers, tying the effects together, lowering our expectations of a non-lambertian effect even further.

However, we recognize that our method of binning pixels together and averaging them all makes us significantly less sensitive to dramatic changes over tiny angular extents, such as the sharp central peak seen in the opposition effect, sometimes less than one degree away from opposition \citep{Kulyk2008, Schaefer2008}. It is worthwhile to perform a more focused check for such an effect.

\subsection{Opposition Effect Search} \label{sec:oppositionC}

We actively looked for opposition effects in the dune data discussed above and found none--these would occur at places where incidence and emission were close to each other, and azimuth was at or near 180$^{\circ}$. We unfortunately had very little data exactly at these points, but if the opposition effect was somewhat broad on Titan, we would have expected to still see some "humps" in the data that would not have been replicated in the models. This was not the case. Forward scattering is not a component of the opposition effect, though it would be similarly placed in the phase functions, just at 0$^{\circ}$ azimuth instead. We also do not observe it in our skewers.

However, we would not have noticed an extremely sharp opposition effect, as it could conceivably only matter at angles extremely close to direct opposition; that is, near 0$^{\circ}$ incidence, 0$^{\circ}$ emission, 180$^{\circ}$ azimuth. Our dunes data only has a handful of points around this region, and it is at the very border of the phase function models, so looking at skewers is rather unhelpful. Instead, we went back to the original VIMS cub files and looked for ones of the dunes that had viewing gometries within 1$^{\circ}$ of opposition. There was precisely one cube in our data set that met this criteria: cub 1574127168 from flyby T37. We then took the data from this cube and plotted its I/F versus the sun-to-spacecraft angle, which is a measure of how close each pixel was to opposition. The result is \textbf{\color{blue}Figure 19\color{black}}. 

\begin{figure*}[htbp]
\includegraphics[scale = 0.6]{OppositionCubeDunes.pdf}
\centering
\caption{\textbf{\color{red} NOT FINAL\color{black}} cub 1574127168 from flyby T37 with dunes pixels separated out and plotted by opposition proximity and I/F in all eight windows. The cub itself is plotted in the center with a color scheme of red 5.00 $\mu$m, green 2.01 $\mu$m, and blue 1.27 $\mu$m. Note that the image is significantly stretched, the actual surface of Titan covered by this image is significant, and greatly extended in the horizontal direction. The gap in the middle of the points exists due to those pixels not being dune pixels. Note that most of the points follow a generally linear trend with the exception of a single spike at around 3$^{\circ}$ from opposition. We can visually see some somewhat brighter dunes pixels in the central image in the lower left, which no doubt cause this spike. The physical location of the pixels closest to true opposition are in the upper left region.}
\label{fig:19}
\end{figure*}

While there does appear to be a spike in \textbf{\color{blue}Figure 19\color{black}}, it happens at around 3$^{\circ}$ and vanishes as we approach actual opposition. We can identify this feature with a slightly brighter section of the dunes in the lower left of the cube, near the central non-dune strip. We must consider the possibility that this could be a pointing error; Cassini is known to be off by a degree at times \citep{Barnes2008}. Due to the stretching of the cube, the brighter section is within one geographical degree of the pixels labeled closest to opposition. We examined other VIMS views of the nearby geography from different flybys and found that the dunes consistently get slightly brighter in that direction, regardless of viewing gometry, so this is most likely a persistent feature and not a very narrow opposition effect. \textbf{\color{red} Do we need to make yet another figure for this? Ta, T10, T35 have the observations needed if we do.\color{black}} 

We must contend with the fact that the opposition effect is all but removed by diffuse lighting and indirect viewing angles from scattering \citep{Schroder2008}, which certainly occur at the lower wavelength windows. However, this is not true at all windows, 5.00 $\mu$m in particular sees significant unimpeded sunlight even when it is near sunset \citep{Barnes2018}. If there were an opposition effect being hidden by the atmosphere, we would be able to see it in 5.00 $\mu$m, but we do not. \textbf{\color{blue}Figure 19\color{black}}'s 5.00 $\mu$m window in fact has the least evidence of a spike out of all windows. 

With the spike explained, we find the increase in brightness with proximity to opposition appears vaguely linear, which is expected \citep{Kulyk2008}. With this we conclude our search, declaring that we observe no opposition effect whatsoever in the dunes. However, we cannot say for certain that there is no opposition effect--this is a single observation and we have no way of knowing if VIMS pointing was in error. The closest point to true opposition is reported as 0.25$^{\circ}$ degrees away; assuming this is correct, even it does not entirely rule out a narrow and sharp opposition effect, as these can be confined to within 0.1$^{\circ}$ \citep{Kulyk2008, Schaefer2008}. 

\section{Summary and Conclusion} \label{sec:conclusion}

There were two primary purposes to this paper: to validate the SRTC++ simulation against real data, and to identify how lambertian Titan's dunes were. 

On the validaiton front, SRTC++ in general gave phase function trends that matched the real data, but is clearly not always getting the correct albedo, as evidenced by situations where the recovered albedo is below 0.0, an impossibility. As these moments primarily occur at short wavelengths, this could be the influence of Rayleigh Scattering. Alternatively, or perhaps additionally, its source could be a mistake in the atmospheric characterization--the deviations along emission skewers imply there's certainly something missing there. 

Despite these caveats, SRTC++ still produces smooth lines that stand alongside the real data in most incidence and azimuth skewers, only with inconsistent albedo results. As such, the simulations are still useful for characterizing Titan's surface, particuarly looking for clear deviations from the lambertian ideal. It is telling, then, that we found almost none. There is no evidence of an opposition effect and no evidence of a forward scattering component. The only potential deviation is the curious dimming and brightening some windows exhibit at high emission angles--but as this effect is azimuthally agnostic, it is most likely an atmospheric effect the model doesn't account for, not a surface effect. 

Which leaves us with a profound result: Titan's dunes are lambertian surfaces, or very close to it. This is opposed to the other icy Saturnian moons which have clear opposition effects \citep{Kulyk2008}. What makes Titan this way? While it is not the focus of this paper to come up with physical explanations, we can speculate that, since Titan is theorized to be covered in organic tholins, they may cover the ice and hide its optical properties from incoming light. We note that radar observations do, in fact, observe an opposition effect in the dunes despite how generally dark they are in such measurements \citep{Neish2010, Wye2011}. This disagreement is not unusual, as the length scales over which radar and VIMS probe are vastly different. 

To be fair, Earth sand is also generally lambertian \citep{Hapke1963}, though unlike Titan, Earth's dune terrains are quite bright. However, even sand on Earth exhibits the opposition effect within two degrees of ideal \citep{Wise2022}, so the fact that we don't see any sign of it is curious. 

Future work on this front would include improving SRTC++ to emulate the atmosphere properly at high emission--or, if that were to prove impossible, investigating what other effects might cause the innacuracies at high emission. This work could also be turned to the specular lakes at Titan's north pole; while the atmosphere is not as well characterized there, we can use the relatively simple lake phase functions to probe where specifically it's poorly characterized. Other terrain types on Titan could also be examined--it would just require a manual rather than automated approach, as the other terrains are not as distinct as the dunes. Preliminary investigations on this front can be found in the appendix.

\begin{acknowledgments}
Acknowledgements:

\textbf{\color{blue}Data availability? Would like to make it clear that we'll give all the information after just being asked... and helpful reviewers as well, but that comes later.\color{black}}

\textbf{\color{red}[Not sure who needs to be put here who won't be put on the author list. Though there is going to be funding recongition here.]\color{black}}
\end{acknowledgments}

\appendix

\section{Equatorial Bright Terrain}

Like the dunes, the equatorial bright terrain (known as "plains" in radar maps) is extensive and has lots of observations over many viewing geometries. Originally we had planned to treat it alongside the dunes and use both to bolster our conclusions. However, as can be seen in \textbf{\color{blue}Figure 20\color{black}}, the equatorial bright terrain was far less well-behaved than the dunes. 

\begin{figure*}[htbp]
\includegraphics[scale = 0.4]{BrightAzimuthSkewer.pdf}
\centering
\caption{Azimuth angle skewer: azimuth angle versus average I/F comparing phase function models with equatorial bright data with fixed incidence and emission angles. Showcases all eight wavelengths arranged from shortest to longest, with the central area occupied by a geometry diagram to illustrate the exact situation plotted. Model lines are monochromatic, equatorial bright data are green. Places with direct equatorial bright observations have dots plotted overtop of the lines, larger dots meaning more observations. Places on the equatorial bright lines without dots are interpolated values. Note that the vertical axis scales with the data, not all wavelengths produce the same average I/F scale.}
\label{fig:20}
\end{figure*}

If the inconsistency in the equatorial bright terrain was predictable, then we would likely have continued to try to draw conclusions from it. However, as we can see in \textbf{\color{blue}Figure 20\color{black}}, the observations have large stretches where the points are in line, before suddenly jumping up or down. This forces us to conclude that what is labeled as equatorial bright terrain is actually multiple different terrain types with their own phase functions. This is not just a distinction between flat areas and mountains--while VIMS has a hard time differentiating those, the radar data used to make the terrain maps had a separate "hummocky" terrain type that was not included in the equatorial bright terrain mask. 

In general, the equatorial bright terrain would follow the lines of the phase function models, and it too exhibits the deviations at high emission. But any other deviations we noted could not have anything else said about them, and even the level to which the terrain followed the models' shapes was suspect. A dedicated investigation will likely need to be devoted to this terrain in the future, perhaps even extending it beyond the equatorial regions if that is deemed reasonable. 

\bibliography{Bibliography}{}
\bibliographystyle{aasjournal}

\end{document}