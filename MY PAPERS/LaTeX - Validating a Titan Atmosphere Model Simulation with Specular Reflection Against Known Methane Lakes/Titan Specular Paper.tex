\documentclass{article}   

%This section is called the preamble
\title{\textbf{Validating a Titan Atmosphere Model Simulation with Specular Reflection Against Known Methane Lakes}}  
\author{Gabriel M Steward\textsuperscript{1}, Jason W. Barnes\textsuperscript{1}, \\Others... \textbf{\color{red}(Will definitely, Shannon?)\color{black}}\\
		{\scriptsize \textsuperscript{1}Department of Physics, University of Idaho, Moscow, Idaho 83844}
}
   


%\usepackage{amsmath} %Not necessary due to mathtools loading it
\usepackage{amssymb} %math
\usepackage{mathtools} %math
\usepackage{csquotes} %autoquotes
\usepackage[style=authoryear,backend=biber]{biblatex} %bibliography
\usepackage{fancyhdr} %headers
\usepackage[dvipsnames]{xcolor} %colors
\usepackage{graphicx} %graphics
\usepackage{siunitx} %SI units
\usepackage{bm} %bold math
\usepackage{cmap} %make PDF searchable
\usepackage{multicol} %multiple columns

\usepackage{hyperref} %internal links
%always last unless otherwise stated

\addbibresource{Bibliography.bib} %For bibliography

\MakeOuterQuote{"} %make quotes automatic

\hypersetup{ colorlinks,
	linkcolor={red!50!black},
	citecolor={blue!50!black},
	urlcolor={blue!80!black}
} %replaces obnoxious hyperlink boxes with neat colored text.
%use the [hidelinks] option in hyperref to get rid of the colors. 


\begin{document}

\pagestyle{fancy}
\fancyhead{}
\fancyhead[C]{STEWARD et al.}
\maketitle

\begin{abstract}
\textbf{\color{red}ABSTRACTION: this will be done last, as we need to know the end from the beginning to properly do it.\color{black}}
\end{abstract}

 
\section{Introduction}
Titan is one of only two bodies in the Solar System with bodies of stable liquid on its surface, the other being Earth (\cite{Hayes2016}). Unlike the seas we are familiar with, the ones on Titan are made primarily of liquid methane (\cite{Mastrogiuseppe2016}). These seas pose a challenge to radiative transfer models of Titan's atmosphere, for they exhibit behavior markedly different from conventional terrain. The vast majority of terrain, even extremely high albedo terrain, such as that on Enceladus (\cite{Li2023}), reflects light in a diffuse or ``lambertian'' manner. Liquids, meanwhile, arrange themselves with such a smooth surface that they can act as mirrors, producing bright ``specular'' reflections in a prefered direction. Direct specular reflections from the sun are a telltale sign that part of Titan's surface is liquid, as no lambertian surface could ever produce them (\cite{Stephan2010}). There are, however, indirect specular reflections as well, produced when sunlight scatters off somewhere in the atmosphere and proceeds to strike a specular surface at the appropriate angle (\cite{Vixie2015}). Thus, specular reflections can alter the character of a surface dramatically from all angles when an atmosphere is present, which is the case on Titan.

Unfortunately, current radiative transfer models of Titan's atmosphere assume a rough, lambertian surface, perhaps with variable albedo (\textbf{give multiple examples}). Yet, the difference between liquid and solid surfaces in radiative transfer is significant; even ignoring the points of direct specular reflection, the distribution of reflected light is distinct. In order to properly model Titan, this difference needs to be accounted for; not only to ensure that our understanding of Titan's surface-atmosphere interaction is accurate, but also to assist in identifying unknown potentially-liquid terrain on Titan that has never had a favorable viewing geometry for direct specular reflections. 

\textbf{\color{Purple} NOTE\color{black}}: mention the ethane simulation that seemed to do nothing in order to deal with accusations of composition bias.

\textbf{\color{red} [Possible figure showing lambertian and specular reflections at Titan's poles?] \color{black}}

In this paper, we demonstrate a specular reflection routine for SRTC++ (Spherical Radiative Transfer in C++), a radiative transfer code tailored to model Titan in the infrared wavelengths available to Cassini's VIMS instrument (). The new routine enables accurate simulation of liquid surfaces on Titan---in fact, as the properties of methane are well known, the accuracy for liquid surfaces is grater than that of the poorly-constrained land of Titan (). To demonstrate this routine, we begin with Methods, describing in brief the code and model we chose. Results examines the direct output of the completed simulation, and Validation compares those results to known lakes and seas on Titan. The implications of our simulation are considered in the Discussion before we end with a Summary and Conclusion.

\section{Methods}
The primary code for our simulation, SRTC++, is described in detail elsewhere (). However, in order to describe the new routine, a brief overview is required. SRTC++ simulates radiative transfer in a Monte Carlo fashion, making it nondeterministic (). Individual ``photon packets'' are launched toward Titan, with the results of every scattering event in the atmosphere determined randomly (). The detector objects in SRTC++ do not detect these ``photon packets'', but rather the result of their scattering events. Each scattering event has a certain probability of going any individual direction; SRTC++ finds the directions that go to the detectors and determines the intensity the detector sees from that event (). Millions of ``photon packets'' are run for a single simulation, with each stattering event updating the detectors until a full picture of Titan forms at each detector.

In SRTC++, the ground is normally modeled as just another scattering event, just one with different probabilities than an atmospheric scatter (). This works well for rough, lambertian surfaces, where the direction is quite random at macro scales. However, specular surfaces do not fit this mold, as light reflecting off of them follows a deterministic behavior.  The new routine takes advantage of this, calculating two different paths to the detector from every scattering event; one that goes directly to the scatter, and one that bounces off a specular surface first. This does leave a hole in the simulation: ``photon packets'' that pass through the atmosphere, strike the surface, and then leave are missed. Fortunately, these missed packets would be exclusively at the point where the specular reflection is brightest and nowhere else, so minimal information about viewing geometry is lost. Furthermore, if recorded, those points would far outshine anything else in the resulting images, making them difficult to parse.

\textbf{\color{red} [FIGURE: diagram of a single scattering event showing both paths to the detector.] \color{black}}

The angle of the second ``photon packet'' path is determined both by the curved geometry of Titan's surface and the index of refraction of liquid methane (). This new path can be ignored if the surface doesn't have liquid at the required location, however we will not make use of this ability in this paper as our chosen model is a global methane ocean. This model does not accurately represent Titan, but it does not have to: when we model a global methane ocean, we see the surface from almost every possible viewing geometry. This will enable us to compare the specular results to a lambertian simulation, quantifying the differences. 

No model of Titan is sufficient without an amosphere. SRTC++ uses (\textbf{whatever SRTC++ uses}) for Titan's atmosphere. The version of SRTC++ used does not account for atmospheric absorption, but such effects are expected to be minimal at the wavelengths we are simulating ().

The lambertian simulation data is taken from (\textbf{the paper that doens't exist}), also created using SRTC++. We made sure that the  parameters matched between the two simulations. There are 72 detectors placed every \qty{5}{\degree} around Titan's equator at a distance of \qty{1e6}{\kilo\meter}. Each detector sees \qty{3500}{\kilo\meter} out from Titan's center, chosen to be significantly larger than Titan's \qty{2575}{\kilo\meter} radius to avoid any edge artifacts. 

\textbf{\color{red} [FIGURE: arrangement of detectors around Titan.] \color{black}}

Each simulation is run at eight different wavelengths that correspond to the eight atmospheric windows, areas of the electromagnetic spectrum that pierce through Titan's atmosphere and allow characterization of the surface (). Simulations could be run at other wavelengths, but they would mask any surface signals and be of minimal use for our current purposes. 

\section{Results}
\textbf{\color{red} [FIGURE: open with the Specular Simulation image. Option: Animate for the journals that allow it, make sure to have multiple non-animating images as well though] \color{black}}

In the above figure we have collected the results at [WAVELENGTHS] mapped to red, green, and blue; the ``titancolor2'' color scheme commonly used in coloring VIMS images (). The most obvious distinction between real images of Titan and this one is the sharp, blue color. This is to be expected: pure methane's index of refraction does not vary significantly through the tested wavelengths (), and so the atmosphere alone determines the color dependence, and since smaller wavelengths scatter more () the image appears bluer. This happens even in the lambertian simulation, as in that one it is assumed that the surface of Titan is white (). However, while we expect the solid surface of Titan to be different from simulation, we do not expect that to be the case with pure liquid methane, so ``blueness'' is a possible identifier of surface liquid. 

The other primary features of the simulation are expected. The bright central area is near the specular point, caused by ``photon packets'' that nearly passed through the atmosphere unhindered, and so did not get scattered far from the ideal path. The circular shape of this feature flattens as it approaches the limb of Titan, which is to be expected from a slanted reflective surface. The limb brigthening effect is due to the phenomenon of total reflection; once the angle of incidence gets large enough, no light passes through the methane-atmosphere barrier and all of it is reflected (). Toward the terminator, Titan appears greener because the photons of the middle wavelengths travel further before scattering, meaning they get further across the terminator (). \textbf{\color{red}[ problem: why don't we see any red at the very edge then? Why is it so faint?]\color{black}} 

The simulation also produces ``eclipse views'' of Titan backlit by the sun, but those show the atmosphere and not the surface, and so are beyond the scope of this paper. There is certainly worthwhile information to be gleaned at a later date, though. 

\textbf{\color{red} [FIGURE: show the raw images across all eight wavelengths] \color{black}}

\color{Green}The 0.93 micron window has the brightest result... The 5 micron window is distinctly different from the others... FILL THIS OUT LATER ONCE YOU EXAMINE THE FIGURE CLOSER\color{black}

\textbf{\color{red} [FIGURE: show the lambertian version of the simulation, with the same structure as the specular figure above] \color{black}}

Direct comparisons between the specular and lambertian simulations reveal a few key differences. First, the lambertian simulation is brighter than the specular everywhere except near the specular point. This was expected, each simulation's Titan is receiving the same amount of energy, but the specular Titan will preferentially focus its light in a single direction, while the lambertian will not, leading the specular point and areas near it to be bright in the specular simulation while everywhere else is relatively dim. 

Both simulations exhibit a ``blue'' color, though it is noticably more intense for the specular case than the lambertian. \textbf{\color{red}[ Hang on, WHY is it more intense? Might it have something to do with the abledos?]\color{black}}

The lambertian simulation also lacks the stark limb brightening effect at low phase as it has no index of refraction. Some brightening can still be seen at other angles, but this is due to the atmosphere, and the specular simulation also showcases this increasing limb brightening with higher phase.

So far, we have only considered qualitative differences between the specular and lambertian geometries. For quantitative analysis, we chose to deconstruct the simulation data by viewing geometry. We took every single simulation pixel that showed the surface (as opposed to the atmosphere) and determined the incidence, emission, and azimuth angles. Any viewing gometries that were hit more than once were added together and then averaged. The result was a database in incidence, emission, azimuth, and wavelength that showed the intensity \textbf{\color{red}[I have GOT to nail down what the exact words and units are that we're simulating]\color{black}} at every possible geometry. We then subtracted the lambertian value from the specular one.

\textbf{\color{red} [FIGURE: specular-lambertian for all wavelengths. Default shows 0 azimuth, but the animating version cycles through azimuth] \color{black}}

The broad behavior of the above figure is expected: we see that lambertian dominates in the negative (blue) areas, which take up most of the viewing geometries, matching what we saw visually. Specular dominantes in the positive (red) areas, which cluster around high emission and incidence angles and places where emission and incidence match. \textbf{\color{red}[Wait, hold on, investigate: the high angles of emission/incidence are due to the index of refraction. The actual spectular point isn't generally there! Look closer.]\color{black}}. Notably, the behavior of the 5 micron is distinctly different than the other windows, with a distinctly different shape and gradient across the azimuth. This is, in general, expected, as 5 microns is a much wider window separated from the other windows by a significant portion of the electromangetic spectrum (). In fact this difference may be supremely helpful, as the behavior of the 5 micron window with respect to the others at different viewing angles could potentially be used as a test to identify liquid bodies. 

\color{Green}In order to identify bodies... challenges from inaccuracy of lambertian versus accuracy of specular... need multiple reference points at different viewing angles... find the most dramatic viewing angle changes... maybe go based on viewing angles of location... genrealize? Lots of unknowns. \color{black}

Of course, actually making use of the differences in specular and lambertian behavior depends on validation. We expect the specular model to be accurate for large bodies of liquid methane, and we have plenty of viewing geometries from the Cassini mission to test.

\section{Validation}
\color{Green}Known lakes... perhaps also land next to the lakes... table of used flybys/locations... visual comparison first, then qualitative... really hope the validation confirms what we have... demonstrate ``identification'' of a lake using the data in the previous section... \color{black}

\textbf{\color{red}[Validation procedure: compare with known lakes. Explain selection process for which images/flybys we used for this (not that I know what this procedure is yet, as we haven't even started this part). Show a visual comparison first, then a quantitative comparison. (Structure: once per flyby used? Once per feature? Or do all visual comparisons and then all quantitative ones?) Compare the quantitative differences and assign some kind of confidence value as to how close our model is to reality. We HOPE that this validation is confirmed. If it is not we presumably need to go back to the drawing board and figure out what went wrong rather than publishing this paper. (if it goes wrong it's possibly a lack of absorption or some other feature.)]]\color{black}}

\section{Discussion}
Lorem ipsum dolor sit amet, consectetur adipiscing elit, sed do eiusmod tempor incididunt ut labore et dolore magna aliqua. Ut enim ad minim veniam, quis nostrud exercitation ullamco laboris nisi ut aliquip ex ea commodo consequat. Duis aute irure dolor in reprehenderit in voluptate velit esse cillum dolore eu fugiat nulla pariatur. Excepteur sint occaecat cupidatat non proident, sunt in culpa qui officia deserunt mollit anim id est laborum.

\textbf{\color{red}[So we don't know what to put here really since we haven't done the full experiment. Most of the observations I currently know of are best put in Results as they are observations about the results directly rather than any real new knowledge. If validation flies, we do have one piece of knowledge: to find lakes look for fully illuminated disks and find the BLUE in Jason Color. Other possible dicsussion points: deviations from reality, noise at five microns, error quantifications if we can get them, other potential signs of lakes.]
\color{black}}

\section{Summary and Conclusion}
Lorem ipsum dolor sit amet, consectetur adipiscing elit, sed do eiusmod tempor incididunt ut labore et dolore magna aliqua. Ut enim ad minim veniam, quis nostrud exercitation ullamco laboris nisi ut aliquip ex ea commodo consequat. Duis aute irure dolor in reprehenderit in voluptate velit esse cillum dolore eu fugiat nulla pariatur. Excepteur sint occaecat cupidatat non proident, sunt in culpa qui officia deserunt mollit anim id est laborum. 

\textbf{\color{red}[Conclude based on how confident we are in the simluation, and summarize points of new science and potential indications of how to identify future lakes. Keep it simple and short, but make sure to include relevant details, such as precise values that can be used to detect lakes. Summary of most important points is simply helpful to readers]]\color{black}}

\section*{Acknowledgements}

\color{Green}Data availability? Would like to make it clear that we'll give all the information after just being asked...\color{black}

\textbf{\color{red}[Not sure who needs to be put here who won't be put on the author list. Though there is going to be funding recongition here.]\color{black}}

\printbibliography

\end{document}